\chapter*{Wstęp}

Celem projektu było zaprojektowanie topologii sieciowej dla organizacji, uwzględniając różne potrzeby i wymagania działów oraz zapewniając wydajność, bezpieczeństwo i skalowalność sieci. Przedstawiona topologia została zaplanowana w taki sposób, aby umożliwić separację ruchu sieciowego, kontrolę dostępu oraz zapewnienie wysokiej dostępności i redundancji. Projekt zakładał również zastosowanie różnych technologii sieciowych, takich jak VLAN, EtherChannel, STP, OSPF, HSRP, IPv6 oraz ACL, aby sprostać wymaganiom organizacji.

Na podstawie dostarczonego diagramu sieciowego, sieć została podzielona na kilka segmentów (VLAN-ów), które są przypisane do różnych działów i funkcji. Każdy segment ma swoje unikalne urządzenia i usługi, co pozwala na lepsze zarządzanie i kontrolę ruchu sieciowego. W projekcie uwzględniono również redundancję połączeń i urządzeń, co zapewnia wysoką dostępność sieci w przypadku awarii.

\subsection*{Liczba Urządzeń w Sieci}

Na podstawie diagramu sieciowego, liczba urządzeń w sieci przedstawia się następująco:

\begin{itemize}
\item Przełączniki (Switch): 21
\item Routery: 6
\item Komputery (PC): 22
\item Serwery: 9
\item Inne urządzenia: 8
\item Access Pointy: 2
\item Inne elementy: 5
\end{itemize}
