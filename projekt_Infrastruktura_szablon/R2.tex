% ********** Rozdział 2 **********
\chapter{Opis Infrastruktury}

Infrastruktura sieci składa się z wielu VLAN-ów, przełączników i routerów skonfigurowanych do zarządzania różnymi działami i funkcjami. VLAN-y są segmentowane w zależności od działu lub funkcji, takich jak administracja, programiści, menedżerowie, ochrona, monitoring, serwery i goście. Każdy VLAN ma przypisane konkretne urządzenia i usługi. Przełączniki są skonfigurowane z EtherChannel dla redundancji i poprawy wydajności, natomiast routery obsługują routing między VLAN-ami i zapewniają dostęp do sieci zewnętrznych.

\section{Lista VLAN-ów}

\begin{table}[htbp]
\centering
\caption{Lista VLAN-ów}
\begin{tabular}{|c|c|c|}
\hline
\textbf{VLAN ID} & \textbf{Nazwa VLAN}            & \textbf{Opis}                              \\ \hline
10               & VLAN\_Admins                   & Zarząd i administracja, komputery i serwery FTP zarządu i administracji \\ \hline
20               & VLAN\_Programmers              & Programiści, komputery programistów, serwer FTP i DHCP \\ \hline
30               & VLAN\_Managers                 & Menedżerowie projektów, komputery, serwer FTP i DHCP \\ \hline
40               & VLAN\_Security                 & Ochrona, komputery ochrony                \\ \hline
50               & VLAN\_Monitoring               & Monitoring, kamery, serwer monitoringu, czujniki \\ \hline
60               & VLAN\_Servers                  & Serwery deweloperskie 1 i 2, wspólny FTP  \\ \hline
70               & VLAN\_Guests                   & Sieć Wi-Fi dla gości na parterze          \\ \hline
80               & VLAN\_Guests2                  & Sieć Wi-Fi dla gości na piętrze           \\ \hline
95               & VLAN\_Black\_hole              & VLAN dla nieużywanych portów (black hole) \\ \hline
\end{tabular}
\end{table}
\newpage

\section{Tabela Dostępu Między VLAN-ami}

Poniższa tabela przedstawia dostępność między poszczególnymi VLAN-ami w sieci. "Pełny dostęp" oznacza, że VLAN źródłowy ma pełny dostęp do zasobów VLAN-u docelowego, "Ograniczony dostęp" oznacza, że dostęp jest ograniczony, a "Brak dostępu" oznacza, że dostęp jest zablokowany.

\begin{table}[htbp]
\centering
\caption{Tabela Dostępu Między VLAN-ami (Część 1)}
\begin{tabular}{|c|c|c|}
\hline
\textbf{VLAN źródłowy} & \textbf{VLAN docelowy} & \textbf{Typ dostępu} \\ \hline
VLAN 10 (Admins)       & VLAN 20 (Programiści)  & Pełny dostęp         \\ \hline
VLAN 10 (Admins)       & VLAN 30 (Menedżerowie) & Pełny dostęp         \\ \hline
VLAN 10 (Admins)       & VLAN 40 (Ochrona)      & Brak dostępu         \\ \hline
VLAN 10 (Admins)       & VLAN 50 (Monitoring)   & Pełny dostęp         \\ \hline
VLAN 10 (Admins)       & VLAN 60 (Serwery)      & Pełny dostęp         \\ \hline
VLAN 10 (Admins)       & VLAN 70 (Goście)       & Brak dostępu         \\ \hline
VLAN 10 (Admins)       & VLAN 80 (Goście2)      & Brak dostępu         \\ \hline
VLAN 10 (Admins)       & VLAN 95 (Black Hole)   & Brak dostępu         \\ \hline
VLAN 20 (Programiści)  & VLAN 10 (Admins)       & Ograniczony dostęp   \\ \hline
VLAN 20 (Programiści)  & VLAN 30 (Menedżerowie) & Ograniczony dostęp   \\ \hline
VLAN 20 (Programiści)  & VLAN 40 (Ochrona)      & Brak dostępu         \\ \hline
VLAN 20 (Programiści)  & VLAN 50 (Monitoring)   & Ograniczony dostęp   \\ \hline
VLAN 20 (Programiści)  & VLAN 60 (Serwery)      & Pełny dostęp         \\ \hline
VLAN 20 (Programiści)  & VLAN 70 (Goście)       & Brak dostępu         \\ \hline
VLAN 20 (Programiści)  & VLAN 80 (Goście2)      & Brak dostępu         \\ \hline
VLAN 20 (Programiści)  & VLAN 95 (Black Hole)   & Brak dostępu         \\ \hline
VLAN 30 (Menedżerowie) & VLAN 10 (Admins)       & Ograniczony dostęp   \\ \hline
VLAN 30 (Menedżerowie) & VLAN 20 (Programiści)  & Ograniczony dostęp   \\ \hline
VLAN 30 (Menedżerowie) & VLAN 40 (Ochrona)      & Brak dostępu         \\ \hline
VLAN 30 (Menedżerowie) & VLAN 50 (Monitoring)   & Ograniczony dostęp   \\ \hline
VLAN 30 (Menedżerowie) & VLAN 60 (Serwery)      & Pełny dostęp         \\ \hline
VLAN 30 (Menedżerowie) & VLAN 70 (Goście)       & Brak dostępu         \\ \hline
VLAN 30 (Menedżerowie) & VLAN 80 (Goście2)      & Brak dostępu         \\ \hline
VLAN 30 (Menedżerowie) & VLAN 95 (Black Hole)   & Brak dostępu         \\ \hline
\end{tabular}
\end{table}

\begin{table}[htbp]
\centering
\caption{Tabela Dostępu Między VLAN-ami (Część 2)}
\begin{tabular}{|c|c|c|}
\hline
\textbf{VLAN źródłowy} & \textbf{VLAN docelowy} & \textbf{Typ dostępu} \\ \hline
VLAN 40 (Ochrona)      & VLAN 10 (Admins)       & Brak dostępu         \\ \hline
VLAN 40 (Ochrona)      & VLAN 20 (Programiści)  & Brak dostępu         \\ \hline
VLAN 40 (Ochrona)      & VLAN 30 (Menedżerowie) & Brak dostępu         \\ \hline
VLAN 40 (Ochrona)      & VLAN 50 (Monitoring)   & Ograniczony dostęp   \\ \hline
VLAN 40 (Ochrona)      & VLAN 60 (Serwery)      & Brak dostępu         \\ \hline
VLAN 40 (Ochrona)      & VLAN 70 (Goście)       & Brak dostępu         \\ \hline
VLAN 40 (Ochrona)      & VLAN 80 (Goście2)      & Brak dostępu         \\ \hline
VLAN 40 (Ochrona)      & VLAN 95 (Black Hole)   & Brak dostępu         \\ \hline
VLAN 50 (Monitoring)   & VLAN 10 (Admins)       & Pełny dostęp         \\ \hline
VLAN 50 (Monitoring)   & VLAN 20 (Programiści)  & Ograniczony dostęp   \\ \hline
VLAN 50 (Monitoring)   & VLAN 30 (Menedżerowie) & Ograniczony dostęp   \\ \hline
VLAN 50 (Monitoring)   & VLAN 40 (Ochrona)      & Ograniczony dostęp   \\ \hline
VLAN 50 (Monitoring)   & VLAN 60 (Serwery)      & Pełny dostęp         \\ \hline
VLAN 50 (Monitoring)   & VLAN 70 (Goście)       & Brak dostępu         \\ \hline
VLAN 50 (Monitoring)   & VLAN 80 (Goście2)      & Brak dostępu         \\ \hline
VLAN 50 (Monitoring)   & VLAN 95 (Black Hole)   & Brak dostępu         \\ \hline
VLAN 60 (Serwery)      & VLAN 10 (Admins)       & Pełny dostęp         \\ \hline
VLAN 60 (Serwery)      & VLAN 20 (Programiści)  & Pełny dostęp         \\ \hline
VLAN 60 (Serwery)      & VLAN 30 (Menedżerowie) & Pełny dostęp         \\ \hline
VLAN 60 (Serwery)      & VLAN 40 (Ochrona)      & Brak dostępu         \\ \hline
VLAN 60 (Serwery)      & VLAN 50 (Monitoring)   & Pełny dostęp         \\ \hline
VLAN 60 (Serwery)      & VLAN 70 (Goście)       & Brak dostępu         \\ \hline
VLAN 60 (Serwery)      & VLAN 80 (Goście2)      & Brak dostępu         \\ \hline
VLAN 60 (Serwery)      & VLAN 95 (Black Hole)   & Brak dostępu         \\ \hline
VLAN 70 (Goście)       & VLAN 10 (Admins)       & Brak dostępu         \\ \hline
VLAN 70 (Goście)       & VLAN 20 (Programiści)  & Brak dostępu         \\ \hline
VLAN 70 (Goście)       & VLAN 30 (Menedżerowie) & Brak dostępu         \\ \hline
VLAN 70 (Goście)       & VLAN 40 (Ochrona)      & Brak dostępu         \\ \hline
VLAN 70 (Goście)       & VLAN 50 (Monitoring)   & Brak dostępu         \\ \hline
VLAN 70 (Goście)       & VLAN 60 (Serwery)      & Brak dostępu         \\ \hline
VLAN 70 (Goście)       & VLAN 80 (Goście2)      & Brak dostępu         \\ \hline
VLAN 70 (Goście)       & VLAN 95 (Black Hole)   & Brak dostępu         \\ \hline
VLAN 80 (Goście2)      & VLAN 10 (Admins)       & Brak dostępu         \\ \hline
VLAN 80 (Goście2)      & VLAN 20 (Programiści)  & Brak dostępu         \\ \hline
VLAN 80 (Goście2)      & VLAN 30 (Menedżerowie) & Brak dostępu         \\ \hline
VLAN 80 (Goście2)      & VLAN 40 (Ochrona)      & Brak dostępu         \\ \hline
VLAN 80 (Goście2)      & VLAN 50 (Monitoring)   & Brak dostępu         \\ \hline
VLAN 80 (Goście2)      & VLAN 60 (Serwery)      & Brak dostępu         \\ \hline
VLAN 80 (Goście2)      & VLAN 70 (Goście)       & Brak dostępu         \\ \hline
VLAN 80 (Goście2)      & VLAN 95 (Black Hole)   & Brak dostępu         \\ \hline
VLAN 95 (Black Hole)   & VLAN 10 (Admins)       & Brak dostępu         \\ \hline
VLAN 95 (Black Hole)   & VLAN 20 (Programiści)  & Brak dostępu         \\ \hline
VLAN 95 (Black Hole)   & VLAN 30 (Menedżerowie) & Brak dostępu         \\ \hline
VLAN 95 (Black Hole)   & VLAN 40 (Ochrona)      & Brak dostępu         \\ \hline
VLAN 95 (Black Hole)   & VLAN 50 (Monitoring)   & Brak dostępu         \\ \hline
VLAN 95 (Black Hole)   & VLAN 60 (Serwery)      & Brak dostępu         \\ \hline
VLAN 95 (Black Hole)   & VLAN 70 (Goście)       & Brak dostępu         \\ \hline
VLAN 95 (Black Hole)   & VLAN 80 (Goście2)      & Brak dostępu         \\ \hline
\end{tabular}
\end{table}
\newpage

\section{Tabela Przełączników z Zastosowanymi Technologiami}

Poniższa tabela przedstawia listę przełączników używanych w sieci oraz technologie, które są na nich zastosowane. EtherChannel jest używany do zwiększenia przepustowości

\begin{table}[htbp]
\centering
\caption{Tabela Przełączników z Zastosowanymi Technologiami}
\begin{tabular}{|c|c|c|}
\hline
\textbf{Przełącznik} & \textbf{Zastosowane Technologie}       & \textbf{EtherChannel} \\ \hline
Przełącznik 1        & VLAN, Spanning-Tree, PortFast, Trunk, EtherChannel & Tak \\ \hline
Przełącznik 2        & VLAN, Spanning-Tree, PortFast, Trunk, EtherChannel & Tak \\ \hline
Przełącznik 3        & VLAN, Spanning-Tree, PortFast, Trunk, EtherChannel & Tak \\ \hline
Przełącznik 4        & VLAN, Spanning-Tree, PortFast, Trunk, EtherChannel & Tak \\ \hline
Przełącznik 5        & VLAN, Spanning-Tree, PortFast, Trunk & Nie \\ \hline
Przełącznik 6        & VLAN, Spanning-Tree, PortFast, Trunk & Nie \\ \hline
Przełącznik 7        & VLAN, Spanning-Tree, PortFast, Trunk & Nie \\ \hline
Przełącznik 8        & VLAN, Spanning-Tree, PortFast, Trunk & Nie \\ \hline
Przełącznik 9        & VLAN, Spanning-Tree, PortFast, Trunk & Nie \\ \hline
Przełącznik 10       & VLAN, Spanning-Tree, PortFast, Trunk & Nie \\ \hline
Przełącznik 11       & VLAN, Spanning-Tree, PortFast, Trunk & Nie \\ \hline
Przełącznik 12       & VLAN, Spanning-Tree, PortFast, Trunk & Nie \\ \hline
Przełącznik 14       & VLAN, Spanning-Tree, PortFast, Trunk & Nie \\ \hline
Przełącznik 15       & VLAN, Spanning-Tree, PortFast, Trunk & Nie \\ \hline
Przełącznik 16       & VLAN, Spanning-Tree, PortFast, Trunk & Nie \\ \hline
Przełącznik 17       & VLAN, Spanning-Tree, PortFast, Trunk & Nie \\ \hline
Przełącznik 18       & VLAN, Spanning-Tree, PortFast, Trunk & Nie \\ \hline
Przełącznik 19       & VLAN, Spanning-Tree, PortFast, Trunk & Nie \\ \hline
Przełącznik 20       & VLAN, Spanning-Tree, PortFast, Trunk & Nie \\ \hline
Przełącznik 21       & VLAN, Spanning-Tree, PortFast, Trunk & Nie \\ \hline
Przełącznik 88       & VLAN, Spanning-Tree, PortFast, Trunk & Nie \\ \hline
\end{tabular}
\end{table}

\section{Tabela Routerów z Zastosowanymi Technologiami}

Poniższa tabela przedstawia listę routerów używanych w sieci oraz technologie, które są na nich zastosowane. Routery obsługują m.in. protokoły routingu OSPF i HSRP, a także routing między VLAN-ami oraz IPv6.

\begin{table}[htbp]
\centering
\caption{Tabela Routerów z Zastosowanymi Technologiami}
\begin{tabular}{|c|c|}
\hline
\textbf{Router} & \textbf{Zastosowane Technologie}             \\ \hline
Router 1        & OSPF, HSRP, Routing między VLAN-ami, IPv6, ACL    \\ \hline
Router 2        & OSPF, HSRP, Routing między VLAN-ami, IPv6, ACL    \\ \hline
Router 3        & OSPF, HSRP, Routing między VLAN-ami, IPv6, ACL    \\ \hline
Router 4        & OSPF, HSRP, Routing między VLAN-ami, IPv6, ACL    \\ \hline
Router ISP1     & OSPF, Routing między VLAN-ami, IPv6, ACL     \\ \hline
Router ISP2     & OSPF, Routing między VLAN-ami, IPv6, ACL     \\ \hline
\end{tabular}
\end{table}

% ********** Koniec rozdziału **********
